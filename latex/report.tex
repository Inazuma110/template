\documentclass[uplatex]{jsarticle}

\usepackage{mylatex}
\usepackage{ap3}
\usepackage{ascmac}

\title{}
\author{}
\date{}

\usepackage[dvipdfmx]{graphicx}
\begin{document}
\maketitle
% --- main content ---

% \section{ソースコード}
% \subsection{Main}
% \lstinputlisting[
%   caption=Main.java,
%   label=srcMain,
%   language=Java
% ]{./Main.java}

% \pgref{srcMain}では...


% --- main content ---
\end{document}
